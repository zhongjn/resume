% !TEX program = xelatex
% This is my resume
% Chinese translation
% by ice1000

\documentclass{resume}

\usepackage{lastpage}
\usepackage{fancyhdr}
\usepackage{linespacing_fix} % disable extra space before next section
\usepackage[fallback]{xeCJK}
\usepackage{tikz}
\usepackage{graphicx}
\usetikzlibrary{calc}

%% \setmainfont[]{SimSun}
%% \setCJKfallbackfamilyfont{rm}{HAN NOM B}
\setCJKmainfont{DengXian}
%% \renewcommand{\thepage}{\Chinese{page}}

\begin{document}
\pagestyle{fancy}
\fancyhf{}
\renewcommand\headrulewidth{0pt}
\cfoot{\thepage\ of \pageref{LastPage}}

\name{钟嘉农}

\basicInfo{
  \email{zhongjn0@foxmail.com} \textperiodcentered\ 
  \phone{(+86) 188-8892-2732} \textperiodcentered\ 
  \github[zhongjn]{https://github.com/zhongjn}
  % \linkedin[billryan8]{https://www.linkedin.com/in/billryan8}
}

\section{\faGraduationCap\ 教育}
\datedsubsection{\textbf{浙江大学}}{2017.9 -- 现在}
  专业:计算机科学与技术,预计毕业日期2021.6。
  
  GPA:4.3/5.0,rank 20\%。


\section{\faUsers\ 经历}
\datedsubsection{\textbf{上半格科技}}{2019.9 -- 现在}
\role{创始人}{商业软件开发}
\begin{itemize}
  \item 组织6人团队,主持了“云闪清单”产品的开发,针对土木行业进行信息化、自动化。
  \item 立足于自然语言处理、信息检索技术,达到自动化的目的。
\end{itemize}

\section{\faGithubAlt\ 项目}

\datedsubsection{\textbf{云闪清单}, C\#}{}
面向土木行业的效率工具
\begin{itemize}
  \item 担任技术负责人。
  \item 自动分析土木企业积累的历史项目数据,基本自动化清单编制的工作,节省大量人力。尤其是造价咨询企业,直接减少了他们30\%的总工作量。
  \item 核心部分利用自然语言、信息检索中的技术,如索引、相似度衡量、聚类、命名实体识别等,完全自主开发。
\end{itemize}

\datedsubsection{\textbf{Gomokuer}, C++, Python}{\url{https://github.com/zhongjn/gomokuer}}
AlphaZero的复现与优化
\begin{itemize}
  \item 在早期(2018.4)复现AlphaZero,在五子棋上取得了人类大师的水平。
  \item 对蒙特卡洛搜索树提出若干优化算法(投机性评估、搜索树DAG化),使搜索效率提升为原来的接近4倍。
\end{itemize}

\datedsubsection{\textbf{Messier 87}, C++, GLSL}{\url{https://github.com/zhongjn/messier87}}
实时黑洞渲染
\begin{itemize}
  \item 模仿《星际穿越》中的卡冈图雅黑洞,利用广义相对论指导的光线追踪,实时绘制黑洞,达成了较好的艺术效果。
\end{itemize}

\datedsubsection{\textbf{MiniSQL}, C++}{\url{https://github.com/zhongjn/minisql}}
小型关系型数据库
\begin{itemize}
  \item 支持SQL子集(增删改查之外,包括嵌套子查询、复杂表达式等)。
  \item 在查询优化上进行了初步的探索,比如索引利用、子查询重写(JOIN)、算子合并等。
\end{itemize}

\datedsubsection{\textbf{ngOS}(开发中), Rust, x86汇编}{}
小型操作系统
\begin{itemize}
  \item 需要在裸金属上重新搭建Rust的基础设施(原本由OS提供的部分,比如内存管理、互斥量等等)。
  \item 目前完成了内存管理,正在着手进程调度。
\end{itemize}


\section{\faCogs\ 技能}

\datedsubsection {\textbf{编程语言}}{}
\begin{itemize}
  \item 尤其熟悉:C\#、C/C++
  \item 较为熟悉:Rust、Java、GLSL、Python、Verilog
\end{itemize}

\datedsubsection {\textbf{编译原理}}{}
\begin{itemize}
  \item 熟悉编译优化技术,尤其是SSA标量优化、面向硬件优化(缓存、调度等)。
  \item 熟悉高级语言虚拟机技术,阅读过一些.NET CLR的源码。
  \item 熟悉常见语法分析技术,编写过C到LLVM IR的编译器。
\end{itemize}

\datedsubsection {\textbf{体系结构(硬件)}}{}
\begin{itemize}
  \item 熟悉CPU原理及设计,熟悉x86,设计过MIPS指令集的流水线CPU。
  \item 熟悉指令级并行、缓存的原理及性能特征。
\end{itemize}


\begin{tikzpicture}[remember picture,overlay]
\node[anchor=south east,inner sep=0pt] at ($(current page.south east)+(-3.3cm,7.8cm)$) {
    \includegraphics[height=0.8in]{blackhole.png}
  };
\end{tikzpicture}
  
\end{document}

